\section{Conventies}

Om overzichtelijke, uniforme en duidelijke code te hebben is het nodig om een set van conventies te volgen.
Deze conventies zijn gebaseerd op Crockords code conventions\cite{ccd}
en de Google Javascript Style Guide\cite{gjsg}.

\subsection{Taalregels}

\subsubsection{Variabelen}

Alle variables worden door middel van ``\lstinline{var}'' gedeclareerd : zonder een ``\lstinline{var}'' worden deze in de globale omgeving geplaatst.

\subsubsection{Statements}

\begin{itemize}
  \item Elke lijn heeft maximum een simpele statement.
  \item ``\lstinline{;}'' wordt altijd gebruikt om het einde van een statement aan te duiden.
\end{itemize}

\subsubsection{Methoden en eigenschap definities}

De verkozen stijl voor methodes is :\\
\begin{lstlisting}
Foo.prototype.bar = function() {
    /* ... */
};
\end{lstlisting}

De verkozen stijl voor andere eigenschappen is om ze in de constructor te initialiseren :\\
\begin{lstlisting}
/** @constructor */
function Foo() {
    this.bar = value;
}
\end{lstlisting}

\subsubsection{Multiline string literals}

Er worden geen strings over meerdere lijnen geschreven :\\
\begin{lstlisting}
var myString = 'A rather long string of English text, an error message \
                actually that just keeps going and going';
\end{lstlisting}

In plaats daarvan worden ze geconcateneerd :\\
\begin{lstlisting}
var myString = 'A rather long string of English text, an error message ' +
    'actually that just keeps going and going';
\end{lstlisting}

\subsection{Stijlregels}

\subsubsection{Benamingen}

Er wordt \textit{camelCase} gebruikt voor het benoemen van variabelen, functies,\ldots (behalve voor de naam van een bestand) :
\begin{itemize}
	\item \lstinline{functionNamesLikeThis}
	\item \lstinline{variableNamesLikeThis}
	\item \lstinline{ClassNamesLikeThis}
	\item \lstinline{EnumNamesLikeThis}
	\item \lstinline{methodNamesLikeThis}
	\item \lstinline{CONSTANT_VALUES_LIKE_THIS}
	\item \lstinline{foo.namespaceNamesLikeThis.bar}
	\item \lstinline{filenameslikethis.js}
\end{itemize}

\subsubsection{Code formattering}

\paragraph{Indentatie}
Omdat ``\lstinline{TAB}'' verschillende resultaten geeft afhankelijk van het gebruikte programma wordt deze niet gebruikt. Indentatie gebeurt met vier spaties.

\paragraph{Gekrulde haakjes}
De gekrulde haakjes worden op dezelfde lijn als wat ze openen gezet :\\
\begin{lstlisting}
if (something) {
    // ...
} else {
    // ...
}
\end{lstlisting}

\paragraph{Spaties}
Er zijn geen spaties tussen linker en rechter haakjes :\\
\begin{lstlisting}
var arr = [1, 2, 3];  // No space after [ or before ].
\end{lstlisting}

Na ``\lstinline{,}'' wordt er altijd een spatie gezet.

\subsection{Databaseregels}

\subsubsection{Benamingen}
Er wordt \textit{snake\_case} gebruikt voor het benoemen van tabellen, kolommen,\ldots Ze worden ook steeds met kleine letters geschreven:
\begin{itemize}
\item account
\item first\_name
\end{itemize}
\subsubsection{Key velden}
\paragraph{Primary key}
Deze worden steeds "id" genoemd. Dit is kort, simpel en makkelijk te onthouden.
\paragraph{Foreign key}
De velden in een \textit{foreign key} tabel moeten een combinatie zijn van de naam van de gerefereerde tabel en van de gerefereerde velden:
\begin{lstlisting}
CREATE TABLE publication_written_by_person (
  publication_id       bigint NOT NULL REFERENCES publication(id),
  person_id     bigint NOT NULL REFERENCES person(id),
  CONSTRAINT publication_written_by_person_pkey PRIMARY KEY (publication_id,
  person_id));
\end{lstlisting}
