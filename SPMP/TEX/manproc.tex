\section {Management proces} 

\subsection{Werkplanning}

\subsection{Controle plan}
\subsubsection{Requirements controle}
Om een overzicht te houden op alle requirements zal er op de site \cite{Website} een requirements dashboard worden opgezet. Deze tabel zal de requirements bijhouden en hun status per iteratie weergeven. Voor elke requirement zullen de belangrijkste statistieken (prioriteit, moeilijkheidsgraad, ...) terug te vinden zijn.

\subsubsection{Planning controle}

Voor de planning maken we gebruik van Teamwork \cite{Teamwork}. Het laat toe om milestones en deadlines vast te leggen voor het hele team. Bovendien is het ook mogelijk voor een enkel teamlid om zichzelf persoonlijke milestones en deadlines geven. Deze gegevens kunnen dan doorgestuurd worden met behulp van een iCal-feed naar de  agenda van dat teamlid. 

\subsubsection{Raportering}

Op het einde van elke iteratie worden documenten, source code en (eventueel) andere artefacten opgeleverd.
Afspraken in verband met deze opleveringen (waarbij N de groepsnummer voorstelt): 

\begin{itemize}
\item Alle documenten en source code (inclusief unit tests) worden per mail afgeleverd als een
enkele zipfile, met als naam seN-iterM, waarbij M het nummer van de iteratie is (voor
eerste versie van documenten geldt M = 0). De aanlevering gebeurt ten laatste voor 9u00 ’s
ochtends op de dag van de deadline\footnote{Zie de kalender \href{tab:kalender})}.
\item Alle documenten en source code worden worden overeenkomstig getagd/gebranchd (seN-iterM)
in de repository.
\item Andere artefacten (zoals executables) worden apart aangeleverd (direct, of via een link, in de
opleveringsmail) en vermelden duidelijk de overeenkomstige iteratie in de bestandsnaam.
\item De mail van de oplevering bevat een bondig overzicht (lijstje) van wat er precies opgeleverd
wordt.
\end{itemize}

Een presentatie duurt een half uur per groep en wordt ingevuld door een aantal sprekers, met 2 sprekers
per groep. Alle groepsleden moeten minimum een keer presenteren. De volgende zaken worden
besproken of gedemonstreerd:

\begin{itemize}
\item een demo van de toegevoegde functionaliteit ten opzichte van de vorige iteratie: gebruik requirements
dashboard, en toon een testrapport (aantal tests, aantal succesvol)
\item analyse van de ontmoette obstakels en de genomen beslissingen
\item bespreking van de functionaliteiten die aan bod zullen komen in de volgende iteratie
\item bespreking van eventuele obstakels, risico’s, etc. in de volgende iteratie
\item overzicht van de architectuur en design van de applicatie
\item bespreking van de statistieken zoals de tijd per taak en per persoon en van de eventuele vertragingen
(plus oplossingen om deze zo klein mogelijk te houden en te vermijden in de toekomst)
\end{itemize}

\subsection{Risico Management Plan}

\subsubsection{Persoonlijk risico's}
Het persoonlijke risico zijn de goede en slechte eigenschappen van elk teamlid. Door deze te kennen, kunnen de taken beter verdeeld worden en kan iedereen op de juiste manier gestimuleerd worden. De analyse is terug te vinden in onderstaande tabel. 
\begin{longtable}{l}
 \textbf{Arno Moonens}  \\
 \hline
  \underline{Positief} \\ 
  Heeft een basis in CSS, HTML en Javascript \\
  Heeft Information Systems gevolgd \\
  \underline{Negatief} \\
  Is stil \\
\\
  \textbf{Mathieu Reymond} \\
  \hline
  \underline{Positief} \\
  Heeft een matige kennis van CSS en HTML \\
  Kan goed met deadlines werken \\
  Heeft Machine Learning en Information Systems gevolgd \\
  Perfectionistisch \\
  \underline{Negatief} \\
  Heeft geen kennis van Javascript \\
  Perfectionistisch \\
  \\
  \textbf{Sam Vervaeck} \\
  \hline 
  \underline{Positief} \\
  Heeft veel ideeën voor het project \\
  Heeft een goede kennis van CSS, HTML en Javascript \\
  Heeft al ervaring met servers \\
  Heeft Machine Learning en Information Systems gevolgd \\
  Perfectionistisch \\
  \underline{Negatief} \\
  Kan zich verliezen in details \\
  Is niet goed met deadlines \\
  Perfectionistisch \\  
   \\
  \textbf{Yannick Verschueren} \\
  \hline
  \underline{Positief} \\
  Kent Javascript \\
  Is goed met deadlines \\
  Is goed met taal \\
  \underline{Negatief} \\
  Kent geen HTML en CSS \\
  Is laks \\
\\
  \textbf{Wout Van Riel} \\
  \hline
  \underline{Positief} \\
  Heeft een basis in HTML en CSS \\
  Heeft managment ervaring \\
  Graphic design \\
  \underline{Negatief} \\
  Kent geen Javascript \\
  Is laks \\
 
\caption{Persoonlijke Risico Analyse}
\label{tab:pra}
\end{longtable}

\subsubsection{Project risico's}

Aan elk risico wordt een waarde meegegeven dat de kans weergeeft dat het probleem zich gaat voordoen tijdens het project en er wordt ook een waarde meegegeven aan de impact die het probleem gaat hebben op het project. De waarden die deze parameters kunnen aannemen liggen tussen één en negen.

\begin{table}[h]
\centering
\begin{tabular}{l|c|c|l}
Risico & Kans & Impact & Oplossing  \\
\hline
 Meningsverschillen & 9 & 4 & Bemiddeling of stemming \\
 Database model staat functies systeem niet toe & 4 & 7 & Database model veranderen \\
 Bugs & 9 & 5 & Verantwoordelijke aanduiden
\end{tabular}
\caption{Project Risico's}
\label{tab:projris}
\end{table}

