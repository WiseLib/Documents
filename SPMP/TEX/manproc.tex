\section {Leidinggevende werkwijze plan} 

\subsection{Start-up plan}
\subsubsection{Personeelsbezetting}
De groep die aan dit project werkt, is samengesteld door professor Raghnild Van Der Straeten \cite{rvdstrae} en haar assistent Jens Nicolay\cite{jnicolay}. In totaal bestaat de groep uit vijf personen en er is geen mogelijkheid om nieuwe werkkrachten in te huren. Er kan altijd een persoon wegvallen uit de groep. Dit kan door persoonlijke redenen of omdat hij niet in de groep kan functioneren. Als iemand uit de groep wil gaan, zal dit intern en met de professor besproken worden.
In de onderstaande tabel staan de functies en de personen die verantwoordelijk en back-up zijn voor deze functies.

\begin{table}[h]
\centering
\begin{tabular}{c|c|c}
\textbf{Rol} & \textbf{Verantwoordelijke} & \textbf{Back-up}  \\
\hline
 Project Manager & Wout Van Riel & Mathieu Reymond \\
 Design Manager & Mathieu Reymond & Yannick Verschueren \\
 Requirements Manager & Mathieu Reymond & Yannick Verschueren \\
 Test Manager & Yannick Verschueren & Arno Moonens \\
 Quality Assurance & Yannick Verschueren & Arno Moonens \\
 Database Manager & Arno Moonens & Sam Vervaeck \\
 Config Manager & Sam Vervaeck & Wout Van Riel \\
 Webmaster & Sam Vervaeck & Wout Van Riel
\end{tabular}
\caption{Rolverdeling}
\label{tab:rolverdeling}
\end{table}

\subsubsection{Plan om middelen te werven}
Er wordt enkel gebruik gemaakt van vrije software. Dit project heeft geen budget om mee te werken en zal dus dienst doen op vrije software. Het is ook een vereiste die is meegegeven met het project. 

\subsubsection{Trainingsplan}
In onderstaande tabel staan de onderdelen waar het personeel in getraind moeten worden. Naast elk onderdeel staat de wijze hoe het personeel opgeleid zal worden.

\begin{table}[h]
\begin{tabular}{l|l}
LaTeX      & Tutorials en documentatie \\
Javascript & Tutorials en documentatie  \\
HTML       & Tutorials en documentatie  \\
CSS        & Tutorials en documentatie 
\end{tabular}
\caption{Trainingstabel}
\label{tab:training}
\end{table}

\subsection{Werkplan}
\subsubsection{Werk activiteiten}
\begin{table}[h]
\begin{tabular}{l|l|l|c|c|c}
\textbf{Activiteit} & \textbf{Verantwoordelijke} & \textbf{Duur} & \textbf{Document} & \textbf{\begin{tabular}[c]{@{}c@{}}Voorafgaand\\ Document\end{tabular}} & \textbf{\begin{tabular}[c]{@{}c@{}}Opvolgend\\ Document\end{tabular}} \\ \hline
Beheer              & Project Manager            & 30u           & SPMP              &                                                                         &                                                                       \\
Kwaliteit           & Quality Assurance          & 25u           & SQAP              &                                                                         &                                                                       \\
Testen              & Test Manager               & 25u           & STD               &                                                                         &                                                                       \\
Design              & Design Manager             & 25u           & SDD               & SRS                                                                     &                                                                       \\
Vereisten           & Requirements Manager       & 25u           & SRS               &                                                                         & SDD                                                                   \\
Configuratie        & Configuration Manager      & 25u           & SCMP              &                                                                         &                                                                      
\end{tabular}
\caption{Werk activiteiten}
\label{tab:werkAct}
\end{table}

\subsubsection{Werkplanning}

De werkplanning is te zien in figuur ~\ref{fig:planning}.

\subsection{Controle plan}
\subsubsection{Requirements controle}
Om een overzicht te houden van alle requirements zal er op de site \cite{Website} een requirements dashboard worden opgezet. Deze tabel zal de requirements bijhouden en hun prioriteit. Als een requirement is afgewerkt, wordt de requirement doorstreept in de tabel. Mogelijke aanpassingen van de requirements worden tussen de requirements manager en de cliënt besproken. Als er een requirement moet veranderen, dan moet de requirements manager dit als punt op de agenda zetten voor de volgende vergadering en past hij het SRS aan.

\subsubsection{Planning controle}

Er zijn algemene milestones en deadlines die tijdens de vergaderingen worden afgesproken. Iedereen is verantwoordelijk voor zijn eigen deadlines. Op elke vergadering vraagt de project manager aan iedereen de stand van zaken. Als er iemand zijn deadline niet kan halen, dan zal er besproken worden of er iemand anders mee kan helpen of dat de deadline kan opgeschoven worden. Mist iemand zijn deadline zonder geldige reden, dan zal de project manager die persoon daar over aanspreken en zal hij het vermelden op de volgende vergadering. De planning wordt met Teamwork \cite{Teamwork} beheerd. 

\subsubsection{Rapportering} 
\paragraph{Documenten en code} 
Op het einde van elke iteratie worden documenten, source code en (eventueel) andere artefacten opgeleverd.
Afspraken in verband met deze opleveringen: 

\begin{itemize}
\item Alle documenten en source code (inclusief unit tests) worden per mail afgeleverd als een
enkele zipfile, met als naam se2-iterM, waarbij M het nummer van de iteratie is (voor
eerste versie van documenten geldt M = 0). De aanlevering gebeurt ten laatste voor 9u00 ’s
ochtends op de dag van de deadline.
\item Alle documenten en source code worden worden overeenkomstig getagd/gebranchd (se2-iterM)
in de repository.
\item Andere artefacten (zoals executables) worden apart aangeleverd (direct, of via een link, in de
opleveringsmail) en vermelden duidelijk de overeenkomstige iteratie in de bestandsnaam.
\item De mail van de oplevering bevat een bondig overzicht (lijstje) van wat er precies opgeleverd
wordt.
\end{itemize}

\paragraph{Presentatie}
Een presentatie duurt een half uur per groep en wordt ingevuld door een aantal sprekers, met 2 sprekers
per groep. Alle groepsleden moeten minimum een keer presenteren. De volgende zaken worden
besproken of gedemonstreerd:

\begin{itemize}
\item een demo van de toegevoegde functionaliteit ten opzichte van de vorige iteratie: gebruik requirements
dashboard, en toon een testrapport (aantal tests, aantal succesvol)
\item analyse van de ontmoette obstakels en de genomen beslissingen
\item bespreking van de functionaliteiten die aan bod zullen komen in de volgende iteratie
\item bespreking van eventuele obstakels, risico’s, etc. in de volgende iteratie
\item overzicht van de architectuur en design van de applicatie
\item bespreking van de statistieken zoals de tijd per taak en per persoon en van de eventuele vertragingen
(plus oplossingen om deze zo klein mogelijk te houden en te vermijden in de toekomst)
\end{itemize}

\subsection{Risico Management Plan}

\paragraph{Eén van de teamleden is ziek of verlaat het team}
\begin{itemize}
\item Kans: 4
\item Impact: 6
\item Prioriteit: 7
\item Oplossing: De Back-up neemt de positie van het teamlid over
\item Verantwoordelijkheid: Project Manager
\end{itemize}

\paragraph{Slechte communicatie binnen de groep}
\begin{itemize}
\item Kans: 6
\item Impact: 5
\item Prioriteit: 7
\item Oplossing: Er worden algemene afspraken gemaakt over de communicatie
\item Verantwoordelijkheid: Project Manager
\end{itemize}

\paragraph{Een deadline wordt gemist}
\begin{itemize}
\item Kans: 2
\item Impact: 9
\item Prioriteit: 7
\item Oplossing: Alle voortgang wordt op github geplaatst, elk teamlid geeft elke week zijn status op de vergadering
\item Verantwoordelijkheid: Project Manager
\end{itemize}

\paragraph{Miscommunicatie tussen de cliënt en het team}
\begin{itemize}
\item Kans: 2
\item Impact: 7
\item Prioriteit: 7
\item Oplossing: De requirements manager houdt contact met de cliënt en bespreekt onduidelijkheden met de cliënt
\item Verantwoordelijkheid: Requirements Manager
\end{itemize}

\paragraph{Plotselinge verandering in de requirements}
\begin{itemize}
\item Kans: 2
\item Impact: 5
\item Prioriteit: 5
\item Oplossing: Proberen de requirement in de bestaande code toe te voegen
\item Verantwoordelijkheid: Requirements Manager
\end{itemize}

\paragraph{Server is tijdelijk offline}
\begin{itemize}
\item Kans: 1
\item Impact: 8
\item Prioriteit: 8
\item Oplossing: Een mirror server opzetten
\item Verantwoordelijkheid: Configuration Manager
\end{itemize}

\paragraph{Onenigheid tussen de teamleden}
\begin{itemize}
\item Kans: 4
\item Impact: 6
\item Prioriteit: 8
\item Oplossing: Bemiddelen tussen de verschillende leden 
\item Verantwoordelijkheid: Project Manager
\end{itemize}

\paragraph{Database model staat de functies van het systeem niet toe}
\begin{itemize}
\item Kans: 4
\item Impact: 7
\item Prioriteit: 8
\item Oplossing: Het database model veranderen
\item Verantwoordelijkheid: Database Manager
\end{itemize}

\paragraph{Merge-conflicten in de master branch in Git}
\begin{itemize}
\item Kans: 9
\item Impact: 3
\item Prioriteit: 7
\item Oplossing: Er wordt een verantwoordelijke aangesteld die de pull requests moet goedkeuren
\item Verantwoordelijkheid: Quality Assurance
\end{itemize}
