\section{Conventions}

To have a clear and understandable code it is necessary to uphold a set of conventions.
Those conventions are based on Crockord's code conventions\footnote{\url{http://javascript.crockford.com/code.html}} 
and the Google Javascript Style Guide\footnote{\url{http://google-styleguide.googlecode.com/svn/trunk/javascriptguide.xml}}.

\subsection{Language rules}

\subsubsection{Variables}

Always declare variables with \lstinline{var} : When you fail to specify var, the variable gets placed in the global context, potentially clobbering existing values.

\subsubsection{Statements}

\begin{itemize}
  \item Every line should have maximum one simple statement.
  \item Always use \lstinline{;}.
\end{itemize}

\subsubsection{Methods and property definitions}

The preferred style for methods is :
\begin{lstlisting}
Foo.prototype.bar = function() {
    /* ... */
};
\end{lstlisting}

The preferred style for other properties is to initialize the field in the constructor :
\begin{lstlisting}
/** @constructor */
function Foo() {
    this.bar = value;
}
\end{lstlisting}

\subsubsection{Multiline string literals}

Don't use multiline string literals :
\begin{lstlisting}
var myString = 'A rather long string of English text, an error message \
                actually that just keeps going and going';
\end{lstlisting}

Use concatenation instead :
\begin{lstlisting}
var myString = 'A rather long string of English text, an error message ' +
    'actually that just keeps going and going';
\end{lstlisting}

\subsection{Style Rules}

\subsubsection{Naming}

Use camelCase (except for file naming) :
\begin{itemize}
	\item \lstinline{functionNamesLikeThis}
	\item \lstinline{variableNamesLikeThis}
	\item \lstinline{ClassNamesLikeThis}
	\item \lstinline{EnumNamesLikeThis}
	\item \lstinline{methodNamesLikeThis}
	\item \lstinline{CONSTANT_VALUES_LIKE_THIS}
	\item \lstinline{foo.namespaceNamesLikeThis.bar}
	\item \lstinline{filenameslikethis.js}
\end{itemize}

\subsubsection{Code formatting}

\paragraph{Indentation}
Indentation is done with 4 spaces. Don't use \lstinline{TAB}.

\paragraph{Curly braces}
Always start your curly braces on the same line as what they're opening :
\begin{lstlisting}
if (something) {
    // ...
} else {
    // ...
}
\end{lstlisting}

\paragraph{Spaces}
Don't put spaces between left and right parenthesis or between brackets :
\begin{lstlisting}
var arr = [1, 2, 3];  // No space after [ or before ].
\end{lstlisting}

Always put a space after a \lstinline{,}.