\section{Introductie}

\subsection {Project Overzicht}

Het project heeft de naam WiseLib meegekregen omdat het een bibliotheek van wijsheid en kennis moet worden op het net.

WiseLib heeft als doel een webtoepassing te zijn waar onderzoekers publicaties op kunnen plaatsen en beheren. Met behulp van deze webtoepassing zullen de onderzoekers overal aan hun publicaties kunnen. Om het nog toegankelijker te maken zal er ook ondersteuning zijn voor de mobiele gebruiker.

\subsection{Project Deliverables}

WiseLib biedt de gebruiker de volgende mogelijkheden aan:

\begin{itemize}
\item Publicaties uploaden en downloaden
\item Beheren van eigen publicaties
\item Andere publicaties consulteren
\item Mogelijkheid om de site mobiel te gebruiken
\end{itemize}

Om die op tijd te kunnen verwezenlijken, zullen de deadlines van de kalender in tabel \ref{tab:kalender} gevolgd worden.

\begin{table}[h]

\begin{tabular}{|l|c|}
\hline
\textbf{Datum} & \textbf{To Do}\\
\hline
 Woensdag 05/11/2014 & inleveren SPMP \\
Woensdag 19/11/2014 & eerste versie documenten \\
Maandag 15/12/2014 & Einde iteratie 1: opleveren code en documenten \\
Vrijdag 19/12/2014 & presentatie \\
\hline \hline
Dinsdag 03/03/2015 & Einde iteratie 2: opleveren code en documenten \\
Woensdag 11/03/2015 & presentatie\\
Maandag 20/04/2015 & Einde iteratie 3: opleveren code en documenten \\
Woensdag 22/04/2015 & presentatie \\
Vrijdag 15/05/2015 & Einde iteratie 4: finale oplevering\\
Woensdag 20/05/2015 & finale presentatie\\
\hline
\end{tabular}

\caption{Kalender}
\label{tab:kalender}
\end{table}

De documenten die telkens mee moeten afgeleverd worden, zijn de volgende:

\begin{itemize}
\item Software Project Management Plan (SPMP)
\item Software Test Plan (STD)
\item Software Requirements Specification (SRS)
\item Software Design Document (SDD)
\item Minutes van alle vergaderingen
\end{itemize}

\subsection{Evolutie van het SPMP}

De project manager kijkt elke week het SPMP na. Als er nood is aan aanpassing dan zal deze de aanpassingen doen en zal hij de revision chart \ref{tab:revchart} aanvullen.

\subsection{Referentiemateriaal}

\begin{thebibliography}{9}
\bibitem{Website} \emph{Team Website} \url{http://wilma.vub.ac.be/~se2_1415/} \\
\bibitem{GitHub} \emph{GitHub Repository} \url{https://github.com/wiselib} \\
\bibitem{Teamwork} \emph{Teamwork} \url{https://wiselib.teamwork.com} \\
\bibitem{Slack} \emph{Slack} \url{https://wiselib.slack.com} \\
\bibitem{Trello} \emph{Trello} \url{https://trello.com/wiselib} \\
\bibitem{Sharelatex} \emph{ShareLaTeX} \url{https://www.sharelatex.com} \\
\bibitem{rvdstrae} \emph{Ragnhild Van Der Straeten} \href{mailto:rvdstrea@vub.ac.be}{rvdstrea@vub.ac.be} \\
\bibitem{jnicolay} \emph{Jens Nicolay} \href{mailto:jnicolay@vub.ac.be}{jnicolay@vub.ac.be} 
\end{thebibliography}

\subsection{Defenities en Acroniemen}

\begin{table}[h]
\centering
\begin{tabular}{c|c}
\textbf{Acroniem} & \textbf{Betekenis} \\
\hline
SPMP & Software Project Management Plan  \\
SRS & Software Requirements Specification \\
STD & Software Test Plan \\
SDD & Software Design Document \\
SQAP & Software Quality Assurance Plan \\
SCMP & Software Configuration Management Plan 
\end{tabular}
\caption{Ancroniemen}
\label{tab:my_label}
\end{table}