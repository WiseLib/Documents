\section{Technische werkwijze plan}

\subsection{Werkwijze model}

Er zal gebruik gemaakt worden van een iteratieve ontwikkeling. Dit model is gekozen vanwege de agenda waar we in verschillende iteraties code moeten afleveren. Het is ook gekozen omdat het gemakkelijker is om requirements per iteratie te verdelen. Zo wordt beslist aan het begin van elke iteratie welke requirements uitgewerkt zullen worden en zal het einde van de iteratie als deadline dienen voor de documenten en code.

\subsection{Methodes, tools en technieken}

\begin{table}[h]
\begin{tabular}{ll}
Javascript & Front-end en bacek-end \\
HTML & Front-end\\
CSS & Front-end \\
LaTeX & Documentatie \\
MariaSQL  & Database            \\
GulpJS    & Taak-automatisatie  \\
GitHub    & Versiebeheersysteem \\
Travis CI & Testen    
\end{tabular}
\caption{Methodes, tools en technieken}
\label{tab:mtt}
\end{table}

%In dit project worden enkel JavaScript, HTML5 en CSS  gebruikt. Marco-talen, zoals onder meer SASS en CoffeeScript, zullen niet aanwezig zijn in dit project. Er wordt wel gebruik gemaakt van open-source softwarebibliotheken en software-frameworks. De tools die gebruikt worden, zijn terug te vinden in het Configuration Management Plan \ref{SCMP} in tabel \ref{tab:tools}.

%Alle documentatie van het project worden op de GitHub \cite{GitHub} van WiseLib geplaatst.

\subsection{Software Documentatie}

Bij elke iteratie moeten er een aantal documenten mee doorgestuurd worden. \newline
De documenten zijn de volgende:

\begin{itemize}
\item Software Project Management Plan (SPMP)
\begin{itemize}
\item Software Quality Assurance Plan (SQAP)
\item Software Configuration Management Plan (SCMP)
\end{itemize}

\item Software Test Plan (STP)
\item Software Requirements Specification (SRS)
\item Software Design Document (SDD)
\item Minutes van alle vergaderingen 
\item Code documentatie 
\end{itemize}

%\subsubsection{Documentatie van Broncode}
%komt in QA

