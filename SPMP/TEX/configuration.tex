
\section{Configuration Management Plan}

Onder de term "Software Configuration Management" verstaan we alle ondersteunende tools en processen die het makkelijker maken om wijzigingen aan de software aan te brengen en bij te houden. \\

Deze sectie legt de rol van de configuration manager verder vast, alsook enkele van de taken die de configuration manager dient te vervullen.

\subsection{Verantwoordelijkheden}

Het is de primaire taak van de configuration manager om alle ondersteunende tools en processen te overzien en te verbeteren.

\subsubsection{Documentatie}

De configuration manager is niet alleen verplicht om al deze processen en tools uit te werken en te verbeteren, maar ook gebruikersdocumentatie te voorzien. De gebruikersdocumentatie moet voor elk teamlid verstaanbaar zijn, maar moet ook niet meer uitgebreid zijn dan nodig is. \\

De primaire bron van informatie zal altijd terug te vinden zijn in dit document. Verdere informatie zal verder worden uitgelegd op GitHub: ofwel in de wiki, ofwel in één of meerdere README-bestanden in de repository zelf. De locatie van de documentatie hangt af van welke plaat het beste erbij past. \\

Andere manieren van uitleg geven, zoals mondeling overleg, zijn ook altijd mogelijk en worden ten zeerste aangeraden.

\subsubsection{Softwarevereisten}

De configuration manager moet trachten om geen beperking te leggen op de programmeeromgeving die door elk teamlid gebruikt wordt. Op die manier is ieder teamlid vrij om zijn eigen vertrouwde omgeving gebruiken, en kan iedereen zo optimaal mogelijk werken. \\

Het is wel mogeljk dat bepaalde tools onbeschikbaar zijn op bepaalde platformen, maar in dat geval is de configuration manager verplicht op een redelijk alternatief aan te bieden op datzelfde platform.

\subsection{Vastgelegde activiteiten}

Wij hebben in dit project gekozen voor een "Separation of Concerns"-aanpak. Dit geldt niet alleen voor de software zelf maar ook bij de tools die worden gebruikt tijdens de ontwikkeling. Als gevolg hiervan is er geen centraal platform, hoewel er wel enkele hulpmiddelen (voornamelijk Slack) instaan om informatie te aggregeren.

\subsubsection{Communicatie}

Alle informele communicatie verloopt via Slack \cite{Slack}. De formele communicatie verloopt via Teamwork.com \cite{Teamwork}. Op die website zullen ook alle deadlines en verslagen komen te staan, zoals al eerder in dit verslag werd besproken. \\

De issues zullen op GitHub \cite{GitHub} komen te staan. Dit maakt het onder meer mogelijk om naar bepaalde wijzigingen te refereren in het issue zelf, of omgekeerd om issues te sluiten bij het doorsturen van een wijziging. \\

Het is de bedoeling om alle services zo goed mogelijk te integreren, zodat de teamleden niet onnodig moeten switchen tussen verschillende tools. Zo is ook de mailinglijst\footnote{se2-1415@wilma.vub.ac.be} geïntegreerd in Slack.

\subsubsection{Source Control}

Een van de belangrijkste ondersteunende middelen voor elk software project is revision control. Revision control maakt het mogelijk voor het team om bij te houden wie welke wijzigingen aan het project heeft gemaakt, op welk moment, en wat de staat van het project was voor de wijziging van kracht ging. \\

De tool (ook "Version Control System" genoemd) die hiervoor wordt gebruikt is door de opdrachtgever vastgelegd. Git\footnote{http://git-scm.com} zal in combinatie met GitHub \cite{GitHub} gebruikt worden om alle code-reviews mee uit te voeren. Het wordt ook van elk teamlid verwacht dat vanaf de development-fase van start gaat alle issues in de tracker op GitHub zelf worden gezet. \\

De stuctuur en lay-out van de branches (het "Branching model") is voor een groot deel gebaseerd op Git-flow \cite{GitFlow}. Dit source control management plan komt ook vaak terug in andere open-source projecten. Alle teamleden hebben er dus baat bij om vertrouwd te zijn met dit systeem, voor wanneer ze in contact komen met externe libraries of frameworks tijdens dit project. \\

Github-flow \cite{GitHubFlow} is een versimpelde versie van dit model. Het leent veel van de concepten van Git-flow, met als voornaamste wijziging dat de \textit{develop} en \textit{master}-branch samengevoegd worden tot één branch. Door deze samenvoeging is er nagenoeg geen verschil tussen de productieversie en de developmentversie. Versies van het project die als stabiel worden aanzien krijgen een bijbehorende "tag" mee. \\

Om de kwaliteit van het ontwikkelproces verder te bevorderen zullen er nog enkele extra scripts\footnote{\url{Git-hooks}{http://git-scm.com/book/en/v2/Customizing-Git-Git-Hooks}} worden gebruikt  die de ontwikkelaar zullen verbieden om wijzigingen te publiceren die niet doorheen alle tests geraken. Dit verplicht iedereen nog meer om de tests goed te onderhouden, en zorgt ervoor dat het project zich nooit in een gebroken staat bevindt.

\subsubsection{Continuous Integration}

Onder continuous integration verstaan we de automatisatie van processen die tussen het aanbrengen van een wijziging aan de code en het \textit{deployen} van de code naar de productieserver liggen. \\

Omdat sommige tools niet cross-platform werken --- en dus bepaalde teamleden niet in staat zijn om sommige tools uit te voeren --- zal een CI-server in werking genomen worden. Die server zal verantwoordelijk zijn om het project na te kijken op fouten, en kan eventueel het project verder uitrollen indien het werd goedgekeurd. \\

Een veelgebruikte service die een deel van deze processen kan automatiseren is Travis CI \cite{TravisCI}. Deze service is vrij beschikbaar voor open-source projecten en integreert nauw samen met GitHub. Wanneer een bepaalde \textit{commit} in GitHub getagd wordt, zal Travis CI alle tests uitvoeren op de laatste versie van het project.

\subsubsection{Streamlinen van ontwikkelproces}

Sommige tools moet de ontwikkelaar herhaaldelijk oproepen. Om de tijd die de ontwikkelaar hierin steekt kan worden verminderd door gebruik te maken van een "Build System". Dit is een volwaardig programma op zich dat instaat om de juiste tools op het juiste moment aan te roepen.

Er zijn twee Build Systems die naar voren komen in een vergelijking: Grunt\footnote{\url{http://gruntjs.com}} en Gulp\footnote{http://gulpjs.com}. Dit project gebruikt Gulp omwille van zijn versimpelde gebruikersinterface en de mogelijkheid om bepaalde taken asynchroon uit te voeren dankzij Orchestrator\footnote{\url{http://orchestratorjs.org}}. De snelle afhandeling van repetitieve taken kan de ontwikkelaars aanzienlijk wat tijd winnen, en zorgt ervoor dat ze zich nog meer op de code kunnen focussen.

\subsubsection{Veranderlijkheid van afspraken}

Aangezien net zoals de software zelf het ook goed mogelijk is dat de ondersteunende tools of processen in de toekomst moeten veranderen, is het altijd mogelijk om wijzigingen aan dit configuration management plan aan te brengen. De enige vereiste is dat het hele team hiermee akkoord gaat nadat de configuration manager zijn motivatie voor de wijziging heeft uitgelegd. \\

Iedereen van het team heeft ook altijd de mogelijkheid om wijzigingen aan het configuration management plan voor te stellen. In dat geval geldt voor het teamlid dezelfde regels als voor de configuration manager.
