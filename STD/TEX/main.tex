\documentclass[a4paper]{article}

\usepackage[english]{babel}
\usepackage[utf8]{inputenc}
\usepackage{hyperref}
\usepackage{longtable}

\setlength{\parindent}{0pt}

\title{Software Test Document}

\begin{document}

\begin{titlepage}
\vspace*{\fill}
\begin{center}
{\Huge Software Test Plan}\\[0.5cm]
{\Large Yannick Verschueren}\\[0.4cm]
November 2014
\end{center}
\vspace*{\fill}
\end{titlepage}

\newpage

Document geschiedenis

\begin{center}
\begin{tabular}{ c | c | c | c}

Versie & Datum & Auteur/co-auteur & Beschrijving \\ \hline

\hline

1 & November 2014 & Yannick Verschueren & Eerste versie \\

2 & December 2014 & Yannick Verschueren & Tweede versie \\

\hline

\end{tabular}
\end{center}

\newpage

\renewcommand\contentsname{Inhoudstafel}
\tableofcontents

\newpage

\section{Introductie}
\subsection{Domein}

Het Software Test Plan behandelt alle methoden, gebruiken en standaarden die van toepassing zijn bij het testen van het "WiseLib" project. Het domein van de testen beschreven in dit plan beschouwt:

\begin{enumerate}
\item Het testen van de functionele vereisten,beschreven in het SRS %Opgelegd door prof
\item Controleren van de gebruikers vereisten %Opgelegd door team
\item Verzekering van code kwaliteit
\item Unit testen en verwijderen van bugs

\end{enumerate}

\subsection{Doel}
Het doel van dit document is om een overzicht te geven van alle strategieën, procedures, activiteiten en taken die instaan voor het testen van het project. Het document volgt de IEEE standaard voor Software en Systeem Test Documentatie.\footnote{\url{http://standards.ieee.org/findstds/standard/829-2008.html}}

\subsection{Objectieven}

\subsubsection{Primair objectief}

Software projecten bestaan uit verschillende onderdelen: software, hardware en interfaces. De testing procedures verder beschreven in dit document behandelen elk onderdeel met variërende mate.

\begin{description}
\item[Software] De geïmplementeerde software is de basis van het project en wordt dus met bijhorende intensiteit getest. Hierdoor wordt een aanzienlijk deel van de tijd gespendeerd aan het testen van de code.

\item[Hardware] Het systeem draait op twee verschillende types hardware. Het cliënt gedeelte werkt op het apparaat van de gebruiker en is dus buiten het bereik van de tester van het systeem. De server waarop het systeem draait is nogmaals niet in het bezit van het team en is dus geen verantwoordelijkheid voor de tester.

\item[Interface] De functionaliteit van de interface wordt door zowel test manager als webmaster getest.

\end{description}

De test procedures zorgen ervoor dat in het systeem:

\begin{enumerate}
\item De functionele vereisten vervult zijn.
\item De gebruikers het systeem correct en zoals beoogd kunnen gebruiken.
Dit houdt in dat de voorgeschreven use-cases op een correcte manier door de applicatie worden afgehandeld.

\end{enumerate}

\subsubsection{Secundair objectief}

Het secundair objectief in het testen van een systeem is het opsporen en behandelen van alle problemen en risico's, deze problemen delen met het team achter het project en het oplossen van deze problemen op een correcte manier.

Hierbij ligt de nadruk op het efficiënt en correct communiceren van problemen tussen leden van het team. Dit document beschrijft alle tools en platformen die gebruikt worden om problemen te melden en op te lossen.

\section{Referenties}

\begin{itemize}
\item Software Configuration Managment Plan
\item Software Design Document
\end{itemize}

\section{Definities, acroniemen en afkortingen}
\label{sec:definitions}

\subsection{Acroniemen}

\begin{description}

\item[SPMP] Software Project Management Plan
\item[SRS] System Requirement Specifiaction
\item[STD] Software Test Document
\item[SDD] Software Design Document

\end{description}

\subsection{Definities}

\begin{description}

\item[Integriteit niveau] is een indicatie van de relatieve belangrijkheid van een software onderdeel tot de stakeholders of opdrachtgevers.
\item[Unit test] methode om software modulen of stukken broncode (units) afzonderlijk te testen.
\item [Integratie test] fase waarin individuele software modules gecombineerd worden en getest worden als een groep.

\item [Traceerbaarheidsmatrix] deze matrix koppelt de test-cases aan de correcte use-cases

\end{description}

\section{Integriteit niveau's}

Elk onderdeel van het systeem heeft een bepaald integriteit niveau (zie definities in sectie~\ref{sec:definitions}). In geval van het falen van een onderdeel zegt de integriteit in hoeverre dit het gehele systeem zal beïnvloeden en hoe belangrijk het oplossen van het probleem is. Hoe hoger het niveau, hoe meer test zullen worden uitgevoerd.

Er bestaan drie niveaus:

\begin{enumerate}

\item De functionele vereisten.(Hoge prioriteit)

\item De gebruikers vereisten of use-cases. (Gemiddelde prioriteit)

\item Aanvullende functionaliteiten. (Lage prioriteit)

\end{enumerate}

De functionele vereisten hebben de hoogste graad van belangrijkheid aangezien zij opgelegd zijn door de opdrachtgevers. Deze functionaliteiten moeten aanwezig zijn in de applicatie en moeten feilloos werken.
De gebruikers vereisten zijn opgelegd door de ontwikkelaars en zijn afgeleid uit de functionele vereisten. Deze vereisten bepalen op welke manier een functionele vereiste zal worden geïmplementeerd. De ontwikkelaars bespreken wat concreet moet bereikt worden door een functionele vereiste en beslissen dan wat de optimale implementatie zal zijn die functionaliteit ondersteund. Deze vereisten hebben een lager niveau aangezien ze niet noodzakelijk zijn voor de opdrachtgevers.
De aanvullende functionaliteiten hebben de laagste graad aangezien zij weinig of geen belang hebben in het correct functioneren van de applicatie.

\section{Test processen}

Het Mocha \footnote{\url{http://mochajs.org/}} framework wordt gebruikt in combinatie met Should.js \footnote{\url{https://github.com/tj/should.js}} als algemeen test platform.\newline

De intensiteitsgraad en strengheid in het uitvoeren en documenteren van test procedures is evenredig met het integriteit niveau. Hoe lager de graad, hoe lager de intensiteit en strengheid van de test procedures.\newline

De test methodologie wordt besproken in het Quality Assurance onderdeel van het SPMP.

\subsection{Unit testing, integratie testing en validatie testen}

Alle unit en integratie tests worden ontworpen en uitgevoerd door de Test Manager. Bij het schrijven van de unit tests kunnen de teamleden het schrijven van de tests versnellen door bij de methodes en functies die zij implementeren, hun eigen tests te schrijven, alhoewel dit niet verplicht is.\newline


\subsubsection{White-box tests}

Zowel de unit tests als de integratie tests zijn white-box tests, zij worden geïplementeerd met kennis van het gehele systeem. Zij testen de interne structuren en werking van het programma.\newline

De Should.js biblitotheek wordt gebruikt bij de unit en integratie tests van de geschreven code.

Het Mocha framework zorgt ervoor dat alle tests op een correcte manier worden uitgevoerd. Een simpel commando voert alle test uit die zich in de "tests" map bevinden. Deze map heeft een indeling gelijkaardig aan de structuur van de applicatie code.

Alle unit en integratie tests bevinden zich op een testing branch van de GitHub van het project. Deze branch wordt up to date gehouden met de meest recente code.

\subsubsection{Black-box tests}

Onder de black-box tests vallen alle test procedures die geen kennis hebben van de interne werking van het systeem. Hun hoofddoel is het testen van de functionaliteiten.
Hieronder vallen:

\begin{description}

\item[Component interface tests] testen de handeling van data tussen verschillende modules van de applicatie
\item[Systeem tests] verifiëren dat de applicatie voldoet aan de vereisten
\item[Acceptatie tests] worden uitgevoerd door de cliënt.

\end{description}

Deze tests maken weinig of geen gebruik van test platformen, maar worden getest door actief gebruik van (een deel van) de applicatie.

\subsection{Test procedures}

Unit tests vormen de basis van het gehele test proces en zullen dus blijvend worden uitgevoerd gedurende de ontwikkeling van het systeem. Dit betekent dat elke methode zijn unit test zal verkrijgen, en deze blijft bestaan tijdens het evolueren van de methode. Telkens wanneer een methode wordt gewijzigd zal de bijhorende test opnieuw worden uitgevoerd. Zo wordt de correcte werking van iedere methode verzekerd en kan men gemakkelijker achterhalen wat er fout loopt bij het falen van een integratie test.
De eerste unit tests worden geschreven voordat het ontwerp, beschreven in het SDD, wordt geïmplementeerd. Elke klasse beschreven in het SDD zal voor de beduidende methodes een test functie bevatten die de correcte werking van de methode zal verzekeren.\newline

Het schrijven van de tests gebeurd met de Should syntax en wordt geplaatst in de test directory. De tester kan de test manueel oproepen of kan gebruik maken van het Mocha framework om alle testen met een commando uit te voeren. Een derde methode van testen is wanneer code naar GitHub (zie SCMP) wordt geduwd. Hierbij wordt de correcte functie opgeroepen en zal afhankelijk van het resultaat van de test de code op GitHub staan. Naast manuele uitvoering van de testen, zal ook bij elke commit op de test branch elke testfile worden uitgevoerd door het gebruikte build systeem (zie SCMP). \newline

Code die door de testen is gecontroleerd, en door de QA Manager is gevalideerd mag op de master branch worden gepusht. Dit is echter een lang proces en maakt dus gebruik van Git's pull request systeem \footnote{https://help.github.com/articles/using-pull-requests/}. \newline

\subsubsection {Problemen in testing}

In het geval van een ontbrekende test, zal beslist worden wat er moet gebeuren afhankelijk van het type van de ontbrekende test. In het geval van een unit test zal de Test Manager ofwel de taak op zich nemen om de test zo snel mogelijk te implementeren, ofwel wordt er beslist dat de persoon die de methode heeft geschreven zelf verantwoordelijk is voor het schrijven van de test.\newline

In geval van een ontbrekende integratie test zal de Test manager gecontacteerd worden en is deze verplicht zo snel mogelijk de test te schrijven, zodat de implementatie van bijhorende modules zo kort mogelijk moet worden stil gelegd.\newline

Als een test faalt wordt de implementatie van de methode gecontroleerd op correctheid, dit aangezien dit de functie van de test is. Indien de methode volgens de implementator correct is, zal de Test Manager de test controleren en indien nodig aanpassen. Dit kan gebeuren wanneer de interface van een methode wordt gewijzigd.

\end{document}