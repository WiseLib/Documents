\section{Algemene Beschrijving}

\subsection{Product perspectief}

Eén van de vereiste functionaliteiten is het opzoeken van publicaties op het web. hierbij wordt gebruik gemaakt van externe websites. Hierdoor wordt het systeem afhankelijk van externe toepassingen. Indien een van de onderstaande websites onbereikbaar wordt, zal een deel van de applicatie niet meer correct functioneren.

\begin{itemize}
\item http://scholar.google.be
\item http://dl.acm.org
\item http://www.mendeley.com
\item http://ieeexplore.ieee.org
\end{itemize}

%Moet volgens de standaard expliciet vermeld worden.

\subsection{Systeem interface}

\subsubsection{Systeem beperkingen}

De applicatie moet in een webbrowser kunnen worden uitgevoerd. Daarnaast moet de applicatie kunnen gebruikt worden op een mobiel apparaat en moet er dus rekening gehouden worden met een beperkt geheugen en verwerkingssnelheid.

\subsection{Product functies}

De applicatie heeft in grote lijnen volgende functies:

%Lijst misschien nog uitbreiden -> specifiekere functies op schrijven

\begin{enumerate}
\item Profiel aanmaken.
\item Papers aan profiel toevoegen.
\item Papers van andere auteurs bekijken.
\item Sociaal netwerk met andere auteurs opstellen.
\item Belangerijke statistieken van papers en auteurs downloaden
\end{enumerate}


\subsection{Gebruikers karakteristieken}

De applicatie is bedoeld voor auteurs en co-auteurs van academische papers en/of thesissen. Hierdoor hebben gebruikers impliciet een academische achtergrond.

De gebruiker heeft voor het gebruik van de applicatie noch ervaring noch enige technische kennis nodig.

Anonieme gebruikers (gebruikers zonder account) kunnen ook gebruik maken van de applicatie, zij het in beperkte mate. 

\subsection{Niet-functionele vereisten / beperkingen}

De applicatie heeft enkele niet-functionele vereisten opgelegd door de opdrachtgevers.
%Copy/Paste van de pdf
\begin{enumerate}
\item GitHub moet gebruikt worden als publieke repository.

\item Er moet een testing framework gekozen en gebruikt worden vanaf de start van de ontwikkeling van de applicatie, en dit gedurende de hele duur van het project. De tests moeten uitgebreid en onderhouden worden bij toevoegingen en wijzigingen van functionaliteit.

\item Enkel JavaScript, HTML5, CSS en bijbehorende open-source frameworks en bibliotheken mogen worden gebruikt als programmeertaal of technologie.

\item Enkel vrije software mag gebruikt worden, zowel voor het eindproduct als voor hulpmiddelen.

\item Het systeem moet eenvoudig kunnen geïnstalleerd worden, op een standaard manier.

\item De grafische gebruikersinterface moet aantrekkelijk en eenvoudig zijn.

\end{enumerate}

De vereisten 2, 5 en 6 worden gecontroleerd door de Quality Assurance en de Test manager. Voor vereiste 3 wordt onderhandeld met de opdrachtgevers. De meetbaarheid van de niet-functionele vereisten hangt dus af van vereiste tot vereiste.

\subsection{Onderstellingen}

De ontwikkelaars van het systeem gaan er van uit dat de gebruiker van zowel de mobiele als niet-mobiele versie van de applicatie over een moderne browser beschikt. Deze browser moet ondersteuning hebben voor HTML5, CSS en Javascript. 