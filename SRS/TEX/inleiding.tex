\section{Inleiding}

\subsection{Doel}

Het doel van dit document is om een uitgebreide beschrijving te geven van de functionaliteit van WiseLib. 

Het geeft een schets van de software om aan de klant te tonen zodat die zijn goedkeuring kan geven. Ook zal het als een leidraad dienen tijdens het ontwikkelen van de software omdat elke feature van WiseLib uitgebreid wordt beschreven in dit document.

\subsection{Scope}

WiseLib is een webtoepassing dat onderzoekers toelaat om publicaties te beheren. Ze kunnen met deze applicatie publicaties uploaden en andere publicaties raadplegen. De applicatie geeft de gebruiker suggesties voor andere publicaties op basis van zijn voorkeuren. Elke publicatie krijgt ook een score toegewezen zodat de gebruiker een top 3 van zijn publicaties kan zien. Een gebruiker is ook in staat een sociaal netwerk op te bouwen aan de hand van zijn publicaties. \newline

Een anonieme gebruiker, zonder account, kan enkel publicaties bekijken. Als hij wil gebruikmaken van de features die WiseLib aanbiedt zoals publicaties uploaden, annotaties maken of een sociaal netwerk genereren, moet hij een account hebben. Een internetconnectie is vereist om informatie van de server te halen. De applicatie zal gratis zijn en zal ook voor mobiele gebruikers toegankelijk zijn. 

