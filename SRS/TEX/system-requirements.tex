\section{System Requirements}

\subsection{Functional Requirements}

\subsubsection{Accountsysteem - ACC}

\begin{longtable}{lp{10cm}}
ID           & ACC1\\
TITLE        & Account Aanmaken \\
PRIORITY     & high \\
DEPENDENCY   & none \\
DESC         & Een gast-gebruiker moet een nieuw account kunnen aanmaken.
Een account heeft volgende velden :
\begin{itemize}
  \item Voornaam
  \item Achternaam
  \item E-mail
  \item Hiërarchische affiliatie
  \item Onderzoeksdomein
\end{itemize}     
\end{longtable}

\begin{longtable}{lp{10cm}}
ID           & ACC2\\
TITLE        & Account aanpassen\\
PRIORITY     & medium\\
DEPENDENCY   & ACC1\\
DESC         & Een ingelogde gebruiker moet zijn gegevens kunnen aanpassen.
Wat aanpasbaar is :
\begin{itemize}
  \item E-mail
  \item Hiërarchische affiliatie
  \item Onderzoeksdomein - Alleen toevoegen
  \item Academische discipline
\end{itemize}     
\end{longtable}

\begin{longtable}{lp{10cm}}
ID           & ACC3\\
TITLE        & Bibliotheek\\
PRIORITY     & medium\\
DEPENDENCY   & ACC1\\
DESC         & Een gebruiker heeft een bibliotheek van papers die hij/zij interessant vindt.     
\end{longtable}

\begin{longtable}{lp{10cm}}
ID           & ACC4\\
TITLE        & Bibliotheek aanpassen\\
PRIORITY     & medium\\
DEPENDENCY   & ACC3\\
DESC         & Een gebruiker kan de papers in zijn bibliotheek bekijken en verwijderen.    
\end{longtable}

\begin{longtable}{lp{10cm}}
ID           & ACC5\\
TITLE        & Eigen publicatielijst\\
PRIORITY     & medium\\
DEPENDENCY   & ACC1, PUB1\\
DESC         & Een gebruiker heeft een lijst van zijn eigen publicaties en kan hun details makkelijk bekijken.  
\end{longtable}

\begin{longtable}{lp{10cm}}
ID           & ACC6\\
TITLE        & Contactpersonen\\
PRIORITY     & medium\\
DEPENDENCY   & ACC1\\
DESC         & Een gebruiker kan zien met welke personen hij al in contact is geweest.
Dit zijn :
\begin{itemize}
\item personen van dezelfde afdeling
\item personen met wie hij in het verleden heeft samengewerkt.
\end{itemize}    
\end{longtable}

\begin{longtable}{lp{10cm}}
ID           & ACC7\\
TITLE        & Statistieken van account\\
PRIORITY     & low\\
DEPENDENCY   & ACC1\\
DESC         & Een gebruiker moet volgende statistieken van zijn account kunnen bekijken:
\begin{itemize}
\item Gemaakte publicaties per periode (dag, maand, jaar,\ldots)
\end{itemize}
\end{longtable}

\begin{longtable}{lp{10cm}}
ID           & ACC8\\
TITLE        & Statistieken van account downloaden\\
PRIORITY     & low\\
DEPENDENCY   & ACC5\\
DESC         & Een gebruiker moet een file kunnen downloaden die alle statistieken van zijn account bevat.     
\end{longtable}

\subsubsection{Publicatiesysteem - PUB}

\begin{longtable}{lp{10cm}}
ID           & PUB1\\
TITLE        & Paper toevoegen\\
PRIORITY     & high\\
DEPENDENCY   & ACC1\\
DESC         & Een gebruiker moet een eigen publicatie kunnen toevoegen. Ofwel gebeurt dit rechtstreeks, ofwel via een andere website. De publicaties moeten ook toegankelijk zijn indien de oorspronkelijke URL van de paper niet meer beschikbaar is.
Een paper heeft de volgende meta-informatie :
\begin{itemize}
\item type (conference proceeding, journal, ...)
\item titel
\item auteurs
\item titel van proceedings of journal
\item bij proceedings :
    \begin{itemize}
    \item editors
    \item naam uitgever
    \item stad
    \end{itemize}
\item bij journals :
    \begin{itemize}
    \item volume
    \item nummer
    \end{itemize}
\item pagina's
\item jaar
\item URL
\item keywords
\end{itemize}      
\end{longtable}

\begin{longtable}{lp{10cm}}
ID           & PUB1.1\\
TITLE        & Toevoegen via PDF\\
PRIORITY     & medium\\
DEPENDENCY   & none\\
DESC         & Een gebruiker kan zijn paper als .pdf uploaden.     
\end{longtable}

\begin{longtable}{lp{10cm}}
ID           & PUB1.2\\
TITLE        & Toevoegen via Bibtex\\
PRIORITY     & medium\\
DEPENDENCY   & none\\
DESC         & Een gebruiker kan zijn paper als een bibtex uploaden.     
\end{longtable}

\begin{longtable}{lp{10cm}}
ID           & PUB1.3\\
TITLE        & Handmatig toevoegen\\
PRIORITY     & high\\
DEPENDENCY   & none\\
DESC         & Een gebruiker kan de meta-informatie van zijn paper handmatig invullen.     
\end{longtable}

\begin{longtable}{lp{10cm}}
ID           & PUB2\\
TITLE        & Ranking\\
PRIORITY     & high\\
DEPENDENCY   & none\\
DESC         & Elke publicatie heeft een rank.
De rank is gebaseerd op :
\begin{itemize}
\item Kwaliteit van publicatie (google conference/journal rankings)
\item Impact van publicatie (referenties naar de publicatie in andere publicaties)
\item Aanwezigheid in bibliotheek van andere gebruikers
\item Aantal publicaties gedeeld door het aantal maanden sinds de eerste publicatie
\end{itemize}     
\end{longtable}

\begin{longtable}{lp{10cm}}
ID           & PUB3\\
TITLE        & Paper opzoeken\\
PRIORITY     & medium\\
DEPENDENCY   & PUB2\\
DESC         & Een gebruiker moet een paper kunnen opzoeken in de bibliotheek van de applicatie zelf of andere databases op het internet. Deze locaties van databases zijn:
\begin{itemize}
\item scholar.google.be
\item dl.acm.org
\item mendeley.com
\item ieeexplore.ieee.org
\end{itemize}     
\end{longtable}

\begin{longtable}{lp{10cm}}
ID           & PUB4\\
TITLE        & Suggestie voor papers\\
PRIORITY     & low\\
DEPENDENCY   & ACC1, PUB3\\
DESC         &  De applicatie raadt papers aan die nog niet in de database van de applicatie zelf aanwezig zijn (maar wel in databases van andere sites). Er worden papers aangeraden op basis van relaties tussen co-auteurs, onderzoeksdomeinen en tags.     
\end{longtable}

\begin{longtable}{lp{10cm}}
ID           & PUB5\\
TITLE        & Negeren van suggesties\\
PRIORITY     & low\\
DEPENDENCY   & PUB4\\
DESC         & Een gebruiker kan aanduiden dat men niet geïnteresseerd is in een paper die door het systeem werd aangeraden. Deze paper wordt dan gedurende 30 dagen niet meer voorgesteld. Als men hierna opnieuw aanduidt dat men niet geïnteresseerd is wordt de paper gedurende 90 dagen niet meer getoond. Na een 3e keer wordt de paper niet meer voorgesteld.     
\end{longtable}

\begin{longtable}{lp{10cm}}
ID           & PUB6\\
TITLE        & Annoteren van paper\\
PRIORITY     & low\\
DEPENDENCY   & ACC1\\
DESC         & Gebruikers kunnen tags en commentaar toevoegen aan een paper     
\end{longtable}

\begin{longtable}{lp{10cm}}
ID           & PUB7\\
TITLE        & Links toevoegen aan paper\\
PRIORITY     & low\\
DEPENDENCY   & ACC1\\
DESC         & Gebruikers kunnen volgende zaken toevoegen aan een paper. Deze bevatten telkens links naar een detailpagina (pagina van een auteur,\ldots)
\begin{itemize}
\item co-auteurs
\item citaties
\item bibliografie
\end{itemize}
Daarnaast kunnen ook deze zaken gelinkt worden aan een paper:
\begin{itemize}
\item slides
\item oefeningen
\item andere documenten
\end{itemize}     
\end{longtable}

\begin{longtable}{lp{10cm}}
ID           & PUB8\\
TITLE        & Statistieken van paper bekijken\\
PRIORITY     & medium\\
DEPENDENCY   & PUB1\\
DESC         & Een gebruiker moet volgende statistieken kunnen bekijken van een specifieke paper:
\begin{itemize}
\item Ranking van de publicatie
\item Aantal citaties
\end{itemize}     
\end{longtable}

\begin{longtable}{lp{10cm}}
ID           & PUB9\\
TITLE        & Statistieken van paper downloaden\\
PRIORITY     & low\\
DEPENDENCY   & PUB8\\
DESC         & Een gebruiker moet een file kunnen downloaden die alle statistieken van de betreffende paper bevat.     
\end{longtable}

\subsubsection{Sociale interactie - SOC}

\begin{longtable}{lp{10cm}}
ID           & SOC1\\
TITLE        & sociaal netwerk\\
PRIORITY     & low\\
DEPENDENCY   & ACC1\\
DESC         & De applicatie kan het sociaal netwerk van een gebruiker genereren :
\begin{itemize}
\item gebruikers en publicaties zijn nodes
\item nodes zijn gelinkt volgens hun relatie (co-auteur, citatie...)
\item kleuren worden gebruikt voor duidelijkheid
\item lijndikte afhankelijk van het belang van de relatie
\end{itemize}     
\end{longtable}

\begin{longtable}{lp{10cm}}
ID           & SOC2\\
TITLE        & uitbreidbaar netwerk\\
PRIORITY     & low\\
DEPENDENCY   & SOC1\\
DESC         & Initieel worden er maar beperkte vertakkingen van nodes getoond, maar nodes kunnen uitgeklapt worden.     
\end{longtable}

\begin{longtable}{lp{10cm}}
ID           & SOC3\\
TITLE        & klikbare nodes\\
PRIORITY     & low\\
DEPENDENCY   & SOC1\\
DESC         & De gebruiker kan op nodes klikken voor extra informatie     
\end{longtable}

\subsubsection{Mobiele gebruiker - MOB}

\begin{longtable}{lp{10cm}}
ID           & MOB1\\
TITLE        & Mobiele interface\\
PRIORITY     & medium\\
DEPENDENCY   & none\\
DESC         & Er moet een aangepaste interface zijn voor iemand die gebruik maakt van een mobiel apparaat (hierna                genoemd de "mobiele gebruiker"). Deze inteface moet dus aangepast zijn aan een kleiner scherm     
\end{longtable}

\begin{longtable}{lp{10cm}}
ID           & MOB2\\
TITLE        & Aangepaste performantie voor mobiele gebruiker\\
PRIORITY     & low\\
DEPENDENCY   & MOB1\\
DESC         & De versie van het systeem voor een mobiele gebruiker moet minder verbruiken maar wel een vlotte ervaring aan de gebruiker bieden.
\end{longtable}

\begin{longtable}{lp{10cm}}
ID           & MOB3\\
TITLE        & Voorziening op netwerkonderbrekingen\\
PRIORITY     & low\\
DEPENDENCY   & MOB1\\
DESC         & De mobiele versie moet erop voorbereid zijn dat onderbrekingen in de communicatie met de server kunnen optreden.
\end{longtable}

\begin{longtable}{lp{10cm}}
ID           & MOB4\\
TITLE        & Gebruik maken van features van een mobiel apparaat\\
PRIORITY     & low\\
DEPENDENCY   & MOB1\\
DESC         & Het systeem maakt gebruik van functionaliteit die enkel beschikbaar is bij een mobiel apparaat, zoals:
\begin{itemize}
\item locatiegegevens
\item camera
\end{itemize}  
\end{longtable}